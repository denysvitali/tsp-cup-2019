\documentclass{article}
\usepackage[a4paper,margin=3cm]{geometry}
\usepackage{xcolor}
\renewcommand{\familydefault}{\sfdefault}

\usepackage{minted}
\usepackage{parskip}
\usepackage{amsmath}
\usepackage{mathtools}

\newminted[code]{java}{autogobble,
breaklines, 
frame=single,
bgcolor=black,
style=monokai,
linenos}

\newminted[tsp]{yaml}{autogobble,
breaklines, 
frame=single,
bgcolor=black,
style=monokai,
linenos}

\newcommand{\mvar}[1]{\textrm{\textit{#1}}}

\setlength\parskip{\baselineskip}

\newcommand{\ic}[1]{\mintinline[style=manni]{java}{#1}}

\begin{document}

\begin{titlepage}
\centering
    {\scshape Scuola Universitaria della Svizzera Italiana (SUPSI)\par}
    \vspace{0.5cm}
    {C08003 - Advanced Algorithms and Optimization\par}
    \vspace{5cm}
    {
    {\fontsize{1.5cm}{1.5cm}\selectfont 19\textsuperscript{th} Algorithms Cup\par}
    \vspace{1cm}
    {\Large Solving the Travelling Salesman Problem (TSP) with Java \par}
    }
    \vspace{8cm}
    {\large Denys Vitali\par}
    \vspace{1cm}
    {\large April 2019}
    \vfill
\end{titlepage}

\pagebreak

\tableofcontents
\pagebreak
\section{Introduzione}
Il presente documento contiene le informazioni riguardanti il mio metodo di
risoluzione del problema del commesso viaggiatore (Travelling salesman problem, TSP).
Questo progetto è un requisito del corso di
\textit{Algoritmi Avanzati ed Ottimizzazione (C08003)}, tenuto presso la 
\textit{Scuola Universitaria Professionale della Svizzera Italiana (SUPSI)} durante il
corso di laurea in ingegneria informatica.
\subsection{Scopo}
Lo scopo del progetto è quello di risolvere 10 problemi di TSP (originati dalla
libreria TSPLIB) forniteci ad inizio corso sfruttando uno o più algoritmi
sviluppati in Java.

\subsection{Requisiti}
Di seguito vengono riportati i requisiti per un corretto svolgimento della coppa:
\begin{itemize}
\item Limite di tempo: massimo 3 minuti per problema,
il tempo è inteso da quando il software inizia a quando finisce.
\item 10 problemi (forniti in un file .zip ad inizio corso)
\item Linguaggio di programmazione: Java
\item Obbligatorio l'utilizzo di Maven, vietato l'utilizzo di librerie esterne
\item Esecuzione replicabile: è necessario salvare eventuali seed e parametri da usare
\item Risoluzione dei problemi: effettuata mediante tests. Deve essere possibile
eseguire un test per risolvere un problema.
\end{itemize}

\subsection{Problema del commesso viaggiatore (TSP)}
Il problema che andremo a risolvere attraverso il nostro
 algoritmo è quello del commesso viaggiatore.
La descrizione può essere formalizzata come segue:
un commesso viaggiatore necessita di visitare $n$ città, al
massimo una volta, partendo da una città di partenza $A$ e tornando
alla stessa dopo averle visitate tutte. Si chiede di trovare
qual è il percorso ottimale (ossia il percorso più breve)
che visiti tutte le città. 

Il problema, di tipo NP-hard, viene espresso matematicamente nel modo seguente:
dato un grafo pesato e completo $G = (V, E, w)$ con $n$ veritci, determinare il
ciclo hamiltoniano con il costo più basso. \\
Data una matrice di incidenza $C$ (dove $c_{i,j} \in \{0,1\}$),
ed una matrice dei pesi $w$,
la funzione obiettivo è la seguente:
\begin{equation}
    \min \ \sum_{i = 0}^n \sum_{j = 0}^n c_{i,j} \cdot w_{i,j}
\end{equation}

Dato che la computazione e verifica di tutte le $n!$ possibilità richiederebbe
un tempo troppo elevato (ed il tempo per la risoluzione di un 
problema è limitato a 3 minuti),
nel nostro progetto ci accontenteremo di una "buona" soluzione,
utilizzando dei metodi euristici.

\pagebreak
\section{Svolgimento del progetto}
\subsection{I/O}
Quale step iniziale e fondamentale è stato necessario sviluppare un parser dei problemi
forniti, in quanto questi venivano forniti in un formato proprietario che sfrutta la
struttura seguente:

\begin{tsp}
NAME: ch130
TYPE: TSP
COMMENT: 130 city problem (Churritz)
DIMENSION: 130
EDGE_WEIGHT_TYPE: EUC_2D
BEST_KNOWN : 6110
NODE_COORD_SECTION
1 334.5909245845 161.7809319139
2 397.6446634067 262.8165330708
(...)
EOF
\end{tsp}

Al fine di sfruttare al meglio il tempo a disposizione per i runs ho deciso di spendere
un po' più del mio tempo a sviluppare un parser che fosse efficiente e non facesse
uso di Regular Expressions. Questo ha portato alla creazione della classe \ic{TSPLoader}.
Una volta caricato un problema (tramite il metodo \ic{parseFile()}) viene restituita
un'istanza della classe \ic{TSPData}. Questa classe contiene i file di problema che
verranno poi utilizzati in qualsiasi algoritmo di risoluzione.

\section{Algoritmi Utilizzati}

\subsection{Genetic Algorithm}
Per l'algoritmo genetico ho fatto uso di una funzione di crossover ampiamente
discussa e rivisitata, ottimizzata per il TSP, chiamata EAX (Edge Assembly
Crossover).

I cicli AB sono selezionati tramite una funzione \ic{EAX.rand()} ed
\ic{EAX.heur()} creando un \mvar{E-Set}.
Successivamente dall'\mvar{E-Set} viene generato un'insieme $E_C$ nel modo seguente: 

\begin{equation}
    E_C := (E_A \cap \bar{D}) \cup (E_B \cap D)
\end{equation}
dove $D$ è l'\mvar{E-Set}.

Alternativamente, ed in modo equivalente, si può definire $E_C$ nel modo seguente:

\begin{equation}
    E_C := (E_A \setminus (\mvar{E-set} \cap E_A)) \cup
    (\mvar{E-Set} \cap E_B)
\end{equation}

\end{document}