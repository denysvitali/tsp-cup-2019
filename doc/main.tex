\documentclass{article}
\usepackage[a4paper,margin=3cm]{geometry}
\usepackage{xcolor}
\renewcommand{\familydefault}{\sfdefault}

\usepackage{minted}
\usepackage{parskip}
\usepackage{amsmath}
\usepackage{mathtools}
\usepackage{hyperref}

\newminted[code]{java}{autogobble,
breaklines, 
frame=single,
bgcolor=black,
style=monokai,
linenos}

\newminted[tsp]{yaml}{autogobble,
breaklines, 
frame=single,
bgcolor=black,
style=monokai,
linenos}

\newcommand{\mvar}[1]{\textrm{\textit{#1}}}
\setlength\parskip{\baselineskip}
\newcommand{\ic}[1]{\mintinline[style=manni]{java}{#1}}

\newenvironment{itm}{
\begin{itemize}
  \setlength{\itemsep}{2pt}
  \setlength{\parskip}{2pt}
}{\end{itemize}}

\begin{document}

\begin{titlepage}
    \centering
    {\scshape Scuola Universitaria della Svizzera Italiana (SUPSI)\par}
    \vspace{0.5cm}
    {C08003 - Advanced Algorithms and Optimization\par}
    \vspace{5cm}
    {
        {\fontsize{1.5cm}{1.5cm}\selectfont 19\textsuperscript{th} Algorithms Cup\par}
        \vspace{1cm}
        {\Large Solving the Travelling Salesman Problem (TSP) with Java \par}
    }
    \vspace{8cm}
    {\large Denys Vitali\par}
    \vspace{1cm}
    {\large April 2019}
    \vfill
\end{titlepage}

\pagebreak

\tableofcontents
\pagebreak
\section{Introduzione}
Il presente documento contiene le informazioni riguardanti il mio metodo di
risoluzione del problema del commesso viaggiatore (Travelling salesman problem, TSP).
Questo progetto è un requisito del corso di
\textit{Algoritmi Avanzati ed Ottimizzazione (C08003)}, tenuto presso la
\textit{Scuola Universitaria Professionale della Svizzera Italiana (SUPSI)} durante il
corso di laurea in ingegneria informatica.
\subsection{Scopo}
Lo scopo del progetto è quello di risolvere 10 problemi di TSP (originati dalla
libreria TSPLIB) forniteci ad inizio corso sfruttando uno o più algoritmi
sviluppati in Java.
Nello specifico, i problemi da risolvere provengono dalla libreria TSPLIB, e
sono i seguenti:
\begin{itm}
    \item ch130
    \item d198
    \item eil76
    \item fl1577
    \item kroA100
    \item lin318
    \item pcb442
    \item pr439
    \item rat783
    \item u1060
\end{itm}

\subsection{Requisiti}
Di seguito vengono riportati i requisiti per un corretto svolgimento della coppa:
\begin{itm}
    \item Limite di tempo: massimo 3 minuti per problema,
    il tempo è inteso da quando il software inizia a quando finisce.
    \item 10 problemi (forniti in un file .zip ad inizio corso)
    \item Linguaggio di programmazione: Java
    \item Obbligatorio l'utilizzo di Maven, vietato l'utilizzo di librerie esterne
    \item Esecuzione replicabile: è necessario salvare eventuali seed e parametri da usare
    \item Risoluzione dei problemi: effettuata mediante tests. Deve essere possibile
    eseguire un test per risolvere un problema.
\end{itm}

\subsection{Problema del commesso viaggiatore (TSP)}
Il problema che andremo a risolvere attraverso il nostro
algoritmo è quello del commesso viaggiatore.
La descrizione può essere formalizzata come segue:
un commesso viaggiatore necessita di visitare $n$ città, al
massimo una volta, partendo da una città di partenza $A$ e tornando
alla stessa dopo averle visitate tutte. Si chiede di trovare
qual è il percorso ottimale (ossia il percorso più breve)
che visiti tutte le città.

Il problema, di tipo NP-hard, viene espresso matematicamente nel modo seguente:
dato un grafo pesato e completo $G = (V, E, w)$ con $n$ vertici, determinare il
ciclo hamiltoniano con il costo minore.

Data una matrice di incidenza $C$ (dove $c_{i,j} \in \{0,1\}$),
ed una matrice dei pesi $w$,
la funzione obiettivo è la seguente:
\begin{equation}
    \min \ \sum_{i = 0}^n \sum_{j = 0}^n c_{i,j} \cdot w_{i,j}
\end{equation}

Dato che la computazione e verifica di tutte le $n!$ possibilità richiederebbe
un tempo troppo elevato (ed il tempo per la risoluzione di un
problema è limitato a 3 minuti),
nel nostro progetto ci accontenteremo di una "buona" soluzione,
utilizzando dei metodi euristici.

\pagebreak
\section{Svolgimento del progetto}
\subsection{I/O}
Quale step iniziale e fondamentale è stato necessario sviluppare un parser dei problemi
forniti, in quanto questi venivano forniti in un formato proprietario che sfrutta la
struttura seguente:

\begin{tsp}
    NAME: ch130
    TYPE: TSP
    COMMENT: 130 city problem (Churritz)
    DIMENSION: 130
    EDGE_WEIGHT_TYPE: EUC_2D
    BEST_KNOWN : 6110
    NODE_COORD_SECTION
    1 334.5909245845 161.7809319139
    2 397.6446634067 262.8165330708
    (...)
    EOF
\end{tsp}

Al fine di sfruttare al meglio il tempo a disposizione per i runs ho deciso di spendere
un po' più del mio tempo a sviluppare un parser che fosse efficiente e non facesse
uso di Regular Expressions. Questo ha portato alla creazione della classe \ic{TSPLoader}.
Una volta caricato un problema (tramite il metodo \ic{parseFile()}) viene restituita
un'istanza della classe \ic{TSPData}. Questa classe contiene i file di problema che
verranno poi utilizzati in qualsiasi algoritmo di risoluzione.

\section{Algoritmi Utilizzati}

\subsection{Costruttivi}
Gli algoritmi di tipo costruttivo sono raggruppati nel package: \texttt{TSP.ra.initial}

\subsubsection{Nearest Neighbour}
L'algoritmo di Nearest Neighbour,
ritorna il percorso generato partendo dal nodo di partenza scelto ed utilizzando
il nodo più vicino nel percorso. Questo algoritmo è utilizzato quale inizializzatore
per gli algoritmi basati su Simulated Annealing.

\subsubsection{Random Nearest Neighbour}
L'algoritmo di Random Nearest Neighbour funziona in modo analogo a quello di Nearest
Neighbour, con la sola differenza che, al momento della scelta del prossimo nodo da
visitare, prende in considerazione più archi e ne sceglie uno in modo casuale.

\pagebreak
\subsection{Ottimizzazione Locale}
Gli algoritmi di ottimizzazione locale sono raggruppati nel package \texttt{TSP.ra.intermediate},
insieme ad alcuni di quelli meta-euristici.
L'idea della suddivisione è data dal fatto che, dato un algoritmo costruttivo ("initial"),
è possibile combinare uno o più ottimizzatori locali ("intermediate" / ILS\footnote{Intermediate Local Search})
con limitazioni di tempo e parametrizzazioni particolari.

È così possibile, grazie all'organizzazione delle classi, eseguire il seguente snippet
in un test per combinare a piacere gli algoritmi implementati:

\begin{figure}[h]
    \begin{code}
        private void fl1577_SA(int seed) throws IOException {
                TSPData data = getProblemData("fl1577");
                TSP tsp = new TSP();
                tsp.init(data);

                Route r = tsp.run(data,
                (new CompositeRoutingAlgorithm())
                .startWith(new RandomNearestNeighbour(seed, data))
                .add(new TwoOpt(data))
                .add(new SimulatedAnnealing(seed)
                .setMode(SimulatedAnnealing.Mode.DoubleBridge))
                );

                validateResult(tsp, r, data);
            }
    \end{code}
    \caption{Esecuzione di fl1577 tramite RNN, 2-opt e Simulated Annealing in modalità Double Bridge}
\end{figure}

\subsubsection{Two Opt}
L'algoritmo di 2-opt (anche chiamato 2-exchange) è stato implementato seguendo
i consigli ed il codice fornito dal \href
{http://www.cs.colostate.edu/~cs314/yr2018sp/more_progress/slides/CS314-S4-Sprint4.pdf}{CS Department of the Colorado State University}.
La mia \href{https://github.com/denysvitali/tsp-cup-2019/blob/master/src/main/java/it/denv/supsi/i3b/advalg/algorithms/TSP/ra/SwappablePath.java#L13-L41}
{prima implementazione} risultava parecchio lenta, in quanto creavo un nuovo array
ed un nuovo oggetto ad ogni 2-Opt swap.

Inoltre, nelle mie versioni precedenti ad ogni iterazione
di 2-Opt calcolavo la lunghezza finale del percorso anziché il guadagno. Ciò comportava
a numerosi calcoli inutili, che sono stati ridotti grazie al miglioramento apportato
dalla regola della diseguaglianza triangolare.

\subsubsection{Three Opt}
Nel mio solver è anche presente un'implementazione di 3-opt,
questa non viene però utilizzata in nessuna delle mie risoluzioni.

\subsubsection{Double Bridge}
Al fine di sfruttare una mossa di perturbazione per l'algoritmo di Simulated Annealing
ho implementato la mossa di Double Bridge (una specifica mossa 4-opt che non è reversibile
da delle mosse 2-opt).

\subsection{Meta euristici}
\subsubsection{Ant Colony Optimization - ACS}
Seguendo alcuni consigli e direttive su alcuni libri e paper relativi all'argomento
\cite{aco} \cite{aco-paper} ho implementato l'algoritmo di Ant Colony System utilizzando i seguenti parametri:

\begin{table}
    \begin{center}
        \begin{tabular}{|l|l|l|}
            \hline
            Parametro & Descrizione & Valore \\
            \hline
            $\alpha$      & Importanza del feromone &  1    \\
            $\beta$       & Importanza dell'euristica & 2    \\
            $\rho$        & Deterioramento del feromone & 0.1  \\
            $\varepsilon$ & Evaporazione del feromone & 0.1  \\
            $q_0$         & Fattore di esplorazione & 0.98 \\
            \hline
        \end{tabular}
    \end{center}
    \caption{Parametri per ACS}
\end{table}
\subsubsection{Genetic Algorithm}
Per l'algoritmo genetico ho fatto uso di una funzione di crossover ampiamente
discussa e rivisitata, ottimizzata per il TSP, chiamata EAX (Edge Assembly
Crossover).

I cicli AB sono selezionati tramite una funzione \ic{EAX.rand()} ed
\ic{EAX.heur()} creando un \mvar{E-Set}.
Successivamente dall'\mvar{E-Set} viene generato un'insieme $E_C$ nel modo seguente:

\begin{equation}
    E_C := (E_A \cap \bar{D}) \cup (E_B \cap D)
\end{equation}
dove $D$ è l'\mvar{E-Set}.

Dopo numerosi tentativi verso la strada del genetico con EAX, ho deciso di
fermarmi in quanto non riuscivo ad ottenere un'implementazione valida.
Purtroppo il poco tempo a mia disposizione ed una scarsa documentazione a riguardo
dell'argomento mi hanno portato ad abbandonare questa strada a poche settimane dalla
consegna. Sono però convinto che questa strada mi avrebbe portato
ad ottimi risultati una volta implementata.

L'algoritmo genetico con un crossover base ed inefficiente è però stata implementata
in \href{https://github.com/denysvitali/tsp-cup-2019/tree/4ce199dcb53d31f20663dbe6c4eaf4bffd1222de}{\texttt{4ce199d}}.
Senza una funzione di crossover efficiente, questo algoritmo porta a scarsi risultati.
\subsubsection{Simulated Annealing}
Sconsolato dai risultati ottenuti tramite ACO / GA + EAX, ho deciso di provare a migliorare
il mio algoritmo di Simulated Annealing.

Inizialmente la funzione (non deterministica) di scheduling della temperatura
era definita nel modo seguente:
\begin{equation}
    T = T_0 \cdot (1 - \dfrac{\text{now} - \text{start}}{\text{max\_runtime}})
    , \quad \text{con} \quad T_0 = 100.0
\end{equation}

dove \textit{now} era un'indicazione del tempo corrente,
 (in ms), \textit{start} il tempo di inizio (in ms) e \textit{max\_runtime}
 il tempo massimo di run (definito come 2 min e 50 secondi).
Sfortunatamente questa funzione non è deterministica in quanto dipende dal numero
di operazioni effettuate in un dato lasso di tempo.
Usando una funzione non deterministica per la generazione del valore di temperatura
ottengo automaticamente un risultato non deterministico per ogni run.

Questa problematica mi ha portato a cambiare la funzione di scheduling della
temperatura nel modo seguente:

\begin{equation}
    T_{k+1} = T_k \cdot \alpha^{i}, \quad \text{con} \quad T_0 = 100.0
\end{equation}

dove $i$ è il numero di iterazioni, ed $\alpha$ un parametro impostato ad $\alpha = 0.98$.
\pagebreak
\section{Esecuzione}
Per eseguire il solver è necessario utilizzare Maven ed utilizzare Java 11.
Una volta clonata la repository (da https://github.com/denysvitali/tsp-cup-2019)
è necessario eseguire il seguente comando:
\begin{code}
    mvn -Dtest=TSPRunnerTest#ch130 test
\end{code}
dove \texttt{ch130} corrisponde al nome del problema da risolvere.

\section{Risultati Ottenuti}
La piattaforma utilizzata quale piattaforma di benchmarking è la seguente:

\begin{tabular}{|l|l|}
    \hline
    \textbf{OS}     & Arch Linux                          \\
    \textbf{Kernel} & \texttt{5.1.0-rc6-mainline x86\_64} \\
    \textbf{CPU}    & Intel Core i7-6700HQ CPU @ 2.60GHz  \\
    \textbf{RAM}    & 32 GB                               \\
    \hline
\end{tabular}



\section{Conclusioni}

\pagebreak
\begin{thebibliography}{9}
    \bibitem{aco}
    Marco Dorigo, Thomas Stützle,
    \textit{Ant Colony Optimization}, 2004.

    \bibitem{aco-paper}
    Marco Dorigo, Luca Maria Gambardella,
    IEEE Transactions On Evolutionary Computation, Vol. 1, No. 1
    \textit{Ant Colony System: A Cooperative Learning Approach to the Traveling Salesman Problem},
    1997.
\end{thebibliography}

\end{document}